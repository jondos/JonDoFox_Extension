\documentclass[a4paper,10pt]{scrartcl}
\usepackage[utf8x]{inputenc}
\usepackage{url}
\usepackage [colorlinks=true,urlcolor=blue,hyperfootnotes=false]{hyperref}

%opening
\title{JonDoFox-Erweiterung}
\author{Georg Koppen}

\begin{document}
  \maketitle
  \section{Vogelperspektive}
  Aus der Vogelperspektive betrachtet besteht JonDoFox aus zwei Teilen: Einem,
  der für das GUI zuständig ist und einem anderen, der davon unabhängig und m
  Wesentlichen für die Schutzmaßnahmen verantwortlich ist.
  Ersterer ist in dem /chrome-Order, letzterer in /components bzw. /modules
  angesiedelt.
  \section{Details - Stufe 1}
    \subsection{GUI}
      Das GUI wird mittels XUL (Markup-Sprache) und JavaScript gebaut und
      betrifft folgende Elemente:
      \begin{itemize}
        \item Toolbarbutton: Der relevante Code befindet sich in jondofox-gui.js
          in den Funktionen openPageNewTab(), openDialogPreferences(),
          setProxyNone(), setCustomProxy() und setProxy() für den XUL-Teil siehe
          das toolbarpalette-Element in jondofox-overlay.xul.
        \item Kontextmenü: Im Kontext-Menü sind zwei verschiedene Features
          untergebracht. Zum einen die Möglichkeit, spezielle Links am Proxy
          vorbei herunterzuladen, zum anderen die Möglichkeit, bequem temporäre
          E-Mail-Adressen zu bekommen. Der relevante Code für das erstere findet
          sich in bypassProxyAndSave(), proxyListAdd() und
          noProxyListAdd(). Für das zweite Feature sind die Methoden im
          jondofox.bloodyVikings-Namensraum und die bindings.xml relevant. Für
          beide ist updateContextMenuEntry() wichtig, um die Features im
          Kontextmenü gegebenenfalls zu verstecken oder sichtbar zu machen.
        \item Statusleiste: Für die Statusleiste (die in neueren
          Firefox-Versionen standardmäßig ausgeblendet ist) gilt das Gleiche wie
          oben unter \textit{Toolbarbutton} gesagt mit dem Unterschied, dass
          hier das statusbar-Element relevant ist.
        \item Optionsdialog: Das ist die zentrale Mögichkeit JonDoFox anzupassen
          (abgesehen von der Änderung von about:config-Einträgen). Der Code
          gefindet sich in /chrome/content/dialogs/prefs-dialogs.js sowie in der
          zugehörigen XUL-Datei.
      \end{itemize}
    \subsection{Module und Komponenten}
      \begin{itemize}
        \item Module:
        \begin{itemize}
          \item Adblocker: Die adblock*.js-Dateien werden gegenwärtig nicht
            genutzt (siehe auch die adBlocking.*-Dateien im GUI-Bereich),
            sondern waren als Vorbereitung für eine eigene, zentrale
            Adblock-Liste gedacht.
          \item Bloody Vikings: Die bloodyVikings*.jsm-Dateien enthalten den
            Code dieses Features, der nichts mit dem GUI zu tun hat.
          \item SSL Observatory: Die ssl-observatory*-Dateien enthalten die
            Kern-Funk\-tio\-na\-li\-tät dieses Features (Whitelist,
            CA-Fingerabdrücke und Zertifikat-Extraktion/-Meldung). Ob
            Zertifikate an die EFF geschickt werden, entscheidet allerdings Code
            in request-observer.js abhängig von den GUI-Einstellungen (siehe
            Zeile 405ff.).
          \item Logger: Der zukünftig erweiterungsweit zu benutzende
            Log-Mechanismus ist in log4moz.js zu finden und wird teilweise schon
            eingesetzt.
          \item Utils: In jdfUtils.jsm sollte nach und nach jegliche
            Funktionalität integriert werden, die von mehreren Code-Teilen
            benötigt wird. Gegenwärtig sind dies Methoden zur Anzeige von
            Dialogen und Methoden zur Behandlung von sprachabhängigen
            Textbausteinen sowie die alte Log-Funktionalität.
          \item SafeCache: safeCache.jsm enthält zwei zentrale
            Sicherheitsfeatures. Zum einen werden Elemente von Drittseiten nicht
            aus dem Cache bezogen (falls diese dort vorhanden sind) und zum
            anderen wird ein Tracking mittels Authorization-Header durch
            Drittseiten unterbunden. Die Einbindung erfolgt ebenfalls in
            request-observer.js
        \end{itemize}
        \item Komponenten:
        \begin{itemize}
          \item About: jondofox-about.js implementiert einen Protokoll-Handler
            für about:jondofox, so dass man bequem (neue) Features etc. anzeigen
            kann. Dazu gehört noch die jondofox-features.xhtml in
            /chrome/content.
          \item PrefMapper: preferences-mapper.js ist dafür da, um die zu
            verändernden Firefox-Einstellungen mit den entsprechenden Werten zu
            versorgen, so wie sie in defaults/preferences/preferences.js gegeben
            sind und um dies wieder rückgängig zu machen, falls JonDoFox z.B.
            deinstalliert wird.
          \item PrefHandler: preferences-handler.js stellt die ganzen Methoden
            bereit, um mit Einstellungen zu arbeiten (ist im Prinzip ``nur'' ein
            Wrapper von Mozillas eigener Komponente, welche die Manipulation von
            Einstellungen zum Gegenstand hat).
          \item ProxyManager: proxy-manager.js sorgt dafür, dass die
            entsprechenden Proxy-Einstellungen (Keiner, JonDo, Tor,
            Benutzerdefiniert, Proxy-Ausnahmen\ldots) auf die relevanten
            Einstellungen im Firefox übertragen werden.
          \item RequestObserver: request-observer.js hat eine Reihe von Aufgaben
            neben der Einbindung von safeCache.jsm und dem SSL Observatory-Code.
            Was die Modifikation der Anfrage an den Server angeht, so werden in
            modifyRequest() die Referer-Spoofing-Logik implementiert sowie die
            nötigen Header gesetzt/entfernt. Dies gilt allen voran für die
            Vermeidung von keep-alive-Headern. examineResponse() enthält neben
            der Steuerung von SSL Observatory-Code noch die wichtige Aufgabe,
            wirklich sicher zu gehen, dass die Verbindung geschlossen wird sowie
            kleinere Features (etwas besserer Schutz vor Tracking mittels
            Authorization-Header in Firefox $<$ 12 sowie verbesserte
            Möglichkeiten, Inhalte am Proxy vorbei herunterzuladen).
          \item JDFManager: Das ist das zentrale Stück Code, dass alle Teile im
            Wesentlichen zusammen hält und für die Initialisierung der
            notwendigen anderen Komponenten verantwortlich ist. Teile von dessen
            Code werden zuallererst ausgeführt, nachdem Firefox gestartet wurde
            (siehe: onUIStartup()). Darüber hinaus nimmt der JDFManager noch
            folgende wichtige Funktionen wahr:
            \begin{itemize}
              \item Einstellungs-Observer: Es werden zentrale Einstellungen
                beobachtet, um gegenenfalls eine Warnung anzeigen zu können,
                wenn der Nutzer versucht, sich in den Fuß zu schießen (siehe:
                nsPref:changed-Topic in observe()).
              \item Add-On-Listener: Es werden die gängigen Ereignisse bzgl. des
                Zustandes der JonDoFox-Erweiterung beobachtet, um z.B. extra
                Code aufzurufen, falls die Erweiterung deinstalliert wird (dann
                werden z.B. die JonDoFox-Einstellungen entfernt).
              \item Extension-Kompat: Es wird überprüft, ob die notwendigen
                Erweiterungen des Profils installiert und aktiv sind bzw. ob
                inkompatible Erweiterungen vorhanden sind (siehe
                checkExtensions() und checkExtensionsFF4()).
              \item Plugins: Es werden alle Plugins (im JonDoFox-Profil bis auf
                Flash) deaktiviert (enforcePluginPref()).
              \item Zeitzonenfake: setTimezone() stellt sicher, dass die
                Zeitzone, die im Browser sichtbar ist, auf UTC gesetzt wird.
              \item User Agent: setUserAgent() ist die zentrale Methode, um
                Anpassungen rund um den User Agent umzusetzen. Dabei wird nicht
                nur der User Agent an den jeweiligen Anonymitätsmodus angepasst,
                sondern es wird auch versucht andere relevante HTTP-Header
                korrekt zu setzen.
              \item MIME-Types: Zum Schutz vor Deanonymisierung durch externe
                Anwendungen gibt es die Methoden firstMimeTypeCheck() und
                getUnknownContentDialog(), correctExternalApplications() sowie
                showUnknownContentTypeWarnings(). Dabei wird zum einen sicher
                gestellt, dass externe Anwendungen nicht automatisch zum Laden
                von Dateien genutzt werden. Zum anderen wird in dem
                entpsrechenden Firefox-Dialog ein roter Hinweis eingebunden,
                dass exterene Anwenwendungen die Anonymität kompromittieren
                können.
              \item Proxyeinstellungen: setProxy() ist dafür zuständig die
                entsprechenden Einstellungen rund um den Proxy-Modus an den
                Proxy-Manager weiterzugeben, der dann die eigentlichen
                Änderungen vornimmt.
            \end{itemize}
        \end{itemize}
      \end{itemize}
  \subsection{Details - Stufe 2 (Zusammenspiel)}
\end{document}
