\documentclass[a4paper,10pt]{scrartcl}
\usepackage[utf8x]{inputenc}
\usepackage{url}
\usepackage [colorlinks=true,urlcolor=blue,hyperfootnotes=false]{hyperref}

%opening
\title{Attacker model of JonDoFox}
\author{Georg Koppen}

\begin{document}
  \maketitle
  \section{Motivation}
    In order to provide a proper security extension we have to define an 
    attacker model first. Otherwise we would not know exactly against whom our
    extension would defend a user and above all how we could implement 
    different features properly. The following of this document is solely 
    dedicated to this task\footnote{This document is heavily inspired by the
    very good Torbutton design document. See: 
    \url{https://www.torproject.org/torbutton/en/design/}}.
  \section{The Attacker}
    The attacker can be located at different positions between the user's 
    browser and the targeting website. This is reflected in section 
    \ref{attacklocation}. While there is just one goal for an adversary, i.e.
    identifiying\footnote{``Identification'' here and 
    througout this documentation means getting the IP address of a user. But 
    even if an attacker succeeds therein it does not necessarily mean the user 
    is identified in reality as this depends on a lot of other contextual 
    information.} a user of JonDoFox, he can try to achieve this in more than one
    way as outlined in section \ref{attackways}. The means available to 
    accomplish the deanonymization are listed in section \ref{attackmeans}.
    \subsection{Location}\label{attacklocation}
      \begin{enumerate}
        \item \textbf{Malicious Websites}: could be a source of attack. That does 
          not necessarily mean that this behavior is intentional. Maybe the 
          websites got hacked or include third party content that is inserted by
	  an attacker.
        \item \textbf{Exit Mixes and Upstream Router} could be controlled by an
          attacker.
        \item \textbf{ISP/Local Network/Upstream Router} could try to correlate
	  JonDonym and non-JonDonym activity trying to deanonyize users.
      \end{enumerate}
    \subsection{Goals}\label{attackways}
      \begin{enumerate}
        \item \textbf{Bypassing proxy settings}: The easiest way to identify a
          user is to let her connect directly, without using JonDonym, to a 
          server controlled by the adversary.
        \item \textbf{Correlation of JonDonym/Non-JonDonym traffic}: If 
          bypassing the proxy is not possible an attacker may try to correlate 
          JonDonym and Non-JonDonym traffic in order to deduce the identity of 
          the former out of the latter.
        \item \textbf{Getting location information}: If neither of the former 
          succeeds the adversary may try to get some general location 
          information that helps him to narrow down the region the user is 
          surfing from.
        \item \textbf{History disclosure}: Knowing which websites a user is/was
          visiting could be helpful irrespective if one of the above mentioned
          means resulted in an identification of the user. The browser history
          could even be a missing link leading to the deanonymization depending
          on the already available information the attacker collected.
        \item \textbf{Anonymity set reduction}: Finally, an attacker could try
          to reduce the anonymity set first in order to make it easier to 
          monitor particular users or even reduce the group so far that he can
	  identify them using information otherwise collected. 
      \end{enumerate}
    \subsection{Means}\label{attackmeans}
      \begin{enumerate}
        \item \textbf{Inserting JavaScript}: Even though JavaScript is not 
          capable of circumventing proxy settings directly it can be used in
          numerous ways to achieve the goals in the afroementioned section. E.g. 
          an adversary could try to register event handlers to correlate
          JonDonym and Non-JonDonym traffic or it can be used to disclose the
	  browser history in Firefox up to version 4\footnote{See: 
	  \url{http://jeremiahgrossman.blogspot.com/2006/08/i-know-where-youve-been.html}.}.
        \item \textbf{Inserting CSS}: Next to JavaScript may the attacker 
          insert CSS code. That may lead as well to history disclosure 
          attacks even if JavaScript is disabled. Furthermore, there may
          as well be a correlation between JonDonym and Non-JonDonym state 
	  via so-called CSS popups (See: 
	  \url{http://meyerweb.com/eric/css/edge/popups/demo2.html}).
      \end{enumerate}
\end{document}
